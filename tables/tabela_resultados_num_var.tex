\begin{table}[H]
\centering
\addtolength{\leftskip}{-0.8cm}
\begin{tabular}{cccccccccc}
\hline
 & \multicolumn{4}{c}{\textbf{Regressão Logística}} & \multicolumn{4}{c}{\textbf{Árvore de Decisão}} &  &  &  \\
 & Acur. & Sens. & Especif.& Nº Var. & Acur. & Sens. & Especif. & Nº Var. &  \\ \hline
\textbf{Cannabis} &  &  &  &  &  &  &  &  & \\
Treino & 77,27\% & 71,41\% & 83,82\% & 6& 75,39\% & 74,02\% & 76,92\% & 4&  \\
Teste &  76,43\% &  75,00\% &  78,03\% & 6&  71,43\% &  71,62\%&  71,21\%& 4 &  \\ \hline
\textbf{Ecstasy} &  &  &  &  &  &  &  &  &  \\
Treino & 76,08\% & 68,65\% & 78,88\% & 6& 73,32\% & 76,20\% & 72,24\%  & 3&  \\
Teste &  77,14\% & 65,79\% & 81,37\% & 6& 74,64\% & 69,34\% & 76,47\% & 3 &  \\ \hline
\textbf{Estimulantes} &  &  &  &  &  &  &  &  &  \\
Treino & 72,74\% & 59,29\% & 79,22\% & 6& 68,86\%& 74,71\% & 66,05\% & 3 &  \\
Teste & 70,82\% & 63,74\% & 74,21\% & 6& 64,77\% & 74,72\% & 60,00\% & 3 &  \\ \hline
 &  &  &  &  &  &  &  &  &  \\
 &  &  &  &  &  &  &  &  & 
\end{tabular}
\caption{Acurácia, sensibilidade, especificidade e número de variáveis do modelo obtido para os métodos de regressão logística e árvore de decisão.}
\label{tabela_resultados_com_var}
\end{table}